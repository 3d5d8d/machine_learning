\documentclass{ltjsarticle}      % ← ltjsarticle!
\usepackage{luatexja-fontspec}   % 日本語フォント読み込み
\setmainjfont{Yu Mincho}         % 自由に変更 OK
\setsansjfont{Yu Gothic}

\begin{document}
\section{機械学習で学ぶPythonの基礎}
 Pythonの入門書は多くありますが, 数値計算やリストの扱いで飽きてしまう人が多く, 読み切れる人は少ないでしょう. そこで本稿では
 機械学習を題材にして, Pythonの基礎を学ぶことを目的とします. 具体的には, 機械学習の基本的な概念やアルゴリズムを学びながら, 
 Pythonの文法やデータ構造を理解していきます. これにより, 最速で Pythonの基礎を習得し, 機械学習の実装に取り組むことができるようになります.
\subsection{早速実践機械学習}
 機械学習とは、コンピュータがデータから学習し、予測や分類を行う技術です。
\subsection{Module, Package, Library}
 Pythonでは、機械学習のための多くのライブラリが用意されています。代表的なものには、NumPy、Pandas、Scikit-learn、TensorFlow、PyTorchなどがあります。
 import numpy as np
 という表現は、NumPyというライブラリをnpという名前でインポートすることを意味します。これにより、NumPyの機能をnpという短い名前で使用できるようになります。
 すなわち, import module名 as あだ名 という形でimportするのですね.
 それではここで, moduleを実際に作ってみましょう. 例えば

 moduleの基本的な構成がわかると
 from module名  import 関数名 as あだ名
 という形も用意に理解できるようになります. これはつまり, from filename import function name as あだ名 が可能ということを意味します.
\subsection{機械学習の種類}
 機械学習には、教師あり学習、教師なし学習、強化学習などの種類があります。
\subsection{機械学習の応用}
 機械学習は、画像認識、自然言語処理、音声認識など、様々な分野で応用されています。
\section{機械学習のアルゴリズム}
 機械学習のアルゴリズムには、回帰分析、決定木、ニューラルネットワークなどがあります。
\subsection{回帰分析}
 回帰分析は、数値データの予測に使用される手法です
\subsection{決定木}
 決定木は、データを分類するためのツリー構造のモデルです。
\subsection{ニューラルネットワーク, CNN}
 ニューラルネットワークは、人間の脳の構造を模したモデルで、深層学習に使用されます。
\end{document}
