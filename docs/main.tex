\documentclass{ltjsarticle}      % ← ltjsarticle!
\usepackage{luatexja-fontspec}   % 日本語フォント読み込み
\usepackage{listings}            % コードハイライト用
\usepackage{xcolor}              % 色指定用
\usepackage{fancyvrb}            % 整形済みテキスト用

\setmainjfont{Yu Mincho}         % 自由に変更 OK
\setsansjfont{Yu Gothic}

% コードスタイル設定
\lstset{
    language=bash,
    basicstyle=\ttfamily\small,
    backgroundcolor=\color{gray!10},
    frame=single,
    framesep=5pt,
    breaklines=true,
    showstringspaces=false,
    commentstyle=\color{green!50!black},
    keywordstyle=\color{blue},
    stringstyle=\color{red}
}

\begin{document}
\section{機械学習で学ぶPythonの基礎}
 Pythonの入門書は多くありますが, 数値計算やリストの扱いで飽きてしまう人が多く, 読み切れる人は少ないでしょう. そこで本稿では
 機械学習を題材にして, Pythonの基礎を学ぶことを目的とします. 具体的には, 機械学習の基本的な概念やアルゴリズムを学びながら, 
 Pythonの文法やデータ構造を理解していきます. これにより, 最速で Pythonの基礎を習得し, 機械学習の実装に取り組むことができるようになります.
\subsection{早速実践機械学習}
 機械学習とは、コンピュータがデータから学習し、予測や分類を行う技術です。
\subsection{Module, Package, Library}
 Pythonでは、機械学習のための多くのライブラリが用意されています。代表的なものには、NumPy、Pandas、Scikit-learn、TensorFlow、PyTorchなどがあります。
 import numpy as np
 という表現は、NumPyというライブラリをnpという名前でインポートすることを意味します。これにより、NumPyの機能をnpという短い名前で使用できるようになります。
 すなわち, import module名 as あだ名 という形でimportするのですね.
 それではここで, moduleを実際に作ってみましょう. 例えば

 moduleの基本的な構成がわかると
 from module名  import 関数名 as あだ名
 という形も用意に理解できるようになります. これはつまり, from filename import function name as あだ名 が可能ということを意味します.
\subsection{機械学習の種類}
 機械学習には、教師あり学習、教師なし学習、強化学習などの種類があります。
\subsection{機械学習の応用}
 機械学習は、画像認識、自然言語処理、音声認識など、様々な分野で応用されています。
\section{機械学習のアルゴリズム}
 機械学習のアルゴリズムには、回帰分析、決定木、ニューラルネットワークなどがあります。
\subsection{回帰分析}
 回帰分析は、数値データの予測に使用される手法です
\subsection{決定木}
 決定木は、データを分類するためのツリー構造のモデルです。
\subsection{ニューラルネットワーク, CNN}
 ニューラルネットワークは、人間の脳の構造を模したモデルで、深層学習に使用されます。

\appendix
\section{Gitによるバージョン管理とプロジェクト構成}

\subsection{Gitの基本概念}
Gitは分散型バージョン管理システムで、ソースコードの変更履歴を追跡・管理するためのツールです。
機械学習プロジェクトでは、コードの変更履歴を管理し、実験結果を再現可能にするために重要な役割を果たします。

\subsection{プロジェクト構造とモジュール化}
大規模な機械学習プロジェクトでは、以下のような構造でコードを整理することが推奨されます:

\begin{verbatim}
MyProject/
├── src/                    # ソースコード
│   ├── models/             # モデル定義
│   ├── data/               # データ処理
│   ├── train/              # 学習スクリプト
│   ├── analysis/           # 分析ツール
│   └── visualization/      # 可視化
├── exefile/                # 実行ファイル
├── models/                 # 保存されたモデル
├── results/                # 実験結果
└── data/                   # データセット
\end{verbatim}

\subsection{Pythonのインポートシステム}
異なるディレクトリのモジュールをインポートする際は、パスの設定が重要です:

\begin{lstlisting}[language=Python]
import sys
import os

# プロジェクトルートをパスに追加
sys.path.append(os.path.join(os.path.dirname(__file__), '..', 'src'))

# モジュールのインポート
from train.mn_cnn_train import train_model
from analysis.mn_cnn_lossf import analyze_loss_landscape
from visualization.mn_cnn_plots import plot_training_results
\end{lstlisting}

\subsection{Gitの基本操作}

\subsubsection{全ファイルの追加とコミット}
プロジェクトの全ファイルをGitに追加する基本的な手順:

\begin{lstlisting}
# 現在の状態を確認
git status

# 全ファイルを追加
git add .
# または
git add -A

# コミット
git commit -m "MNIST CNN プロジェクト初期実装"

# リモートリポジトリにプッシュ
git push
\end{lstlisting}

\subsubsection{.gitignoreファイルの活用}
GitHubにアップロードしたくないファイル(大きなデータセットやモデルファイル)は、
\texttt{.gitignore}ファイルで除外します:

\begin{lstlisting}
# Python関連
__pycache__/
*.pyc
*.pyo
.Python

# 機械学習関連
models/*.pt
models/*.pth
data/
results/
*.pkl

# IDE設定
.vscode/
.idea/

# OS関連
.DS_Store
Thumbs.db
\end{lstlisting}

\subsection{改行コードの問題}
異なるOS間でプロジェクトを共有する際、改行コードの違いによる警告が表示されることがあります:

\begin{lstlisting}
warning: in the working copy of 'file.py', 
LF will be replaced by CRLF the next time Git touches it
\end{lstlisting}

この警告は以下の方法で対処できます:

\begin{lstlisting}
# 改行コード自動変換を無効化
git config core.autocrlf false

# またはグローバル設定
git config --global core.autocrlf false
\end{lstlisting}

\subsection{ローカルGitとGitHubの使い分け}
\begin{itemize}
    \item \textbf{ローカルGit}: 全ての変更履歴、大きなファイル、個人的な実験結果を管理
    \item \textbf{GitHub}: ソースコードのみを公開、データセットやモデルファイルは除外
\end{itemize}

この使い分けにより、コードの共有と個人データの保護を両立できます。

\subsection{環境管理の将来展望}

\subsubsection{Docker による環境コンテナ化}
長期的な開発では、Dockerによる環境のコンテナ化が重要になります:

\begin{lstlisting}[language=bash]
# Dockerfile例
FROM python:3.11-slim
WORKDIR /app
COPY requirements.txt .
RUN pip install -r requirements.txt
COPY . .
CMD ["python", "exefile/mn_cnn_main.py"]
\end{lstlisting}

\subsubsection{段階的な環境管理戦略}
\begin{enumerate}
    \item \textbf{学習期}: Git + .gitignore による基本管理
    \item \textbf{中級期}: Docker導入による環境統一
    \item \textbf{上級期}: Kubernetes + CI/CD による本格運用
\end{enumerate}

この段階的アプローチにより、学習コストを最小化しながら
本格的な機械学習システムへの発展が可能になります。

\end{document}
