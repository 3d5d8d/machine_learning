\documentclass{ltjsarticle}      % ← ltjsarticle!
\usepackage{luatexja-fontspec}   % 日本語フォント読み込み
\usepackage{listings}            % コードハイライト用
\usepackage{xcolor}              % 色指定用
\usepackage{fancyvrb}            % 整形済みテキスト用

\setmainjfont{Yu Mincho}         % 自由に変更 OK
\setsansjfont{Yu Gothic}

% コードスタイル設定
\lstset{
    language=bash,
    basicstyle=\ttfamily\small,
    backgroundcolor=\color{gray!10},
    frame=single,
    framesep=5pt,
    breaklines=true,
    showstringspaces=false,
    commentstyle=\color{green!50!black},
    keywordstyle=\color{blue},
    stringstyle=\color{red}
}

\begin{document}
\section{機械学習で学ぶPythonの基礎}
 Pythonの入門書は多くありますが, 数値計算やリストの扱いで飽きてしまう人が多く, 読み切れる人は少ないでしょう. そこで本稿では
 機械学習を題材にして, Pythonの基礎を学ぶことを目的とします. 具体的には, 機械学習の基本的な概念やアルゴリズムを学びながら, 
 Pythonの文法やデータ構造を理解していきます. これにより, 最速で Pythonの基礎を習得し, 機械学習の実装に取り組むことができるようになります.
\subsection{早速実践機械学習}
 機械学習とは、コンピュータがデータから学習し、予測や分類を行う技術です。
\subsection{Module, Package, Library}
 Pythonでは、機械学習のための多くのライブラリが用意されています。代表的なものには、NumPy、Pandas、Scikit-learn、TensorFlow、PyTorchなどがあります。
 import numpy as np
 という表現は、NumPyというライブラリをnpという名前でインポートすることを意味します。これにより、NumPyの機能をnpという短い名前で使用できるようになります。
 すなわち, import module名 as あだ名 という形でimportするのですね.
 それではここで, moduleを実際に作ってみましょう. 例えば

 moduleの基本的な構成がわかると
 from module名  import 関数名 as あだ名
 という形も用意に理解できるようになります. これはつまり, from filename import function name as あだ名 が可能ということを意味します.
\subsection{機械学習の種類}
 機械学習には、教師あり学習、教師なし学習、強化学習などの種類があります。
\subsection{機械学習の応用}
 機械学習は、画像認識、自然言語処理、音声認識など、様々な分野で応用されています。
\section{機械学習のアルゴリズム}
 機械学習のアルゴリズムには、回帰分析、決定木、ニューラルネットワークなどがあります。
\subsection{回帰分析}
 回帰分析は、数値データの予測に使用される手法です
\subsection{決定木}
 決定木は、データを分類するためのツリー構造のモデルです。
\subsection{ニューラルネットワーク, CNN}
 ニューラルネットワークは、人間の脳の構造を模したモデルで、深層学習に使用されます。

\appendix
\section{Gitによるバージョン管理とプロジェクト構成}

\subsection{Gitの基本概念}
Gitは分散型バージョン管理システムで、ソースコードの変更履歴を追跡・管理するためのツールです。
機械学習プロジェクトでは、コードの変更履歴を管理し、実験結果を再現可能にするために重要な役割を果たします。

\subsection{プロジェクト構造とモジュール化}
大規模な機械学習プロジェクトでは、以下のような構造でコードを整理することが推奨されます:

\begin{verbatim}
MyProject/
├── src/                    # ソースコード
│   ├── models/             # モデル定義
│   ├── data/               # データ処理
│   ├── train/              # 学習スクリプト
│   ├── analysis/           # 分析ツール
│   └── visualization/      # 可視化
├── exefile/                # 実行ファイル
├── models/                 # 保存されたモデル
├── results/                # 実験結果
└── data/                   # データセット
\end{verbatim}

\subsection{Pythonのインポートシステム}
異なるディレクトリのモジュールをインポートする際は、パスの設定が重要です:

\begin{lstlisting}[language=Python]
import sys
import os

# プロジェクトルートをパスに追加
sys.path.append(os.path.join(os.path.dirname(__file__), '..', 'src'))

# モジュールのインポート
from train.mn_cnn_train import train_model
from analysis.mn_cnn_lossf import analyze_loss_landscape
from visualization.mn_cnn_plots import plot_training_results
\end{lstlisting}

\subsection{Gitの基本操作}

\subsubsection{全ファイルの追加とコミット}
プロジェクトの全ファイルをGitに追加する基本的な手順:

\begin{lstlisting}
# 現在の状態を確認
git status

# 全ファイルを追加
git add .
# または
git add -A

# コミット
git commit -m "MNIST CNN プロジェクト初期実装"

# リモートリポジトリにプッシュ
git push
\end{lstlisting}

\subsubsection{.gitignoreファイルの活用}
GitHubにアップロードしたくないファイル(大きなデータセットやモデルファイル)は、
\texttt{.gitignore}ファイルで除外します:

\begin{lstlisting}
# Python関連
__pycache__/
*.pyc
*.pyo
.Python

# 機械学習関連
models/*.pt
models/*.pth
data/
results/
*.pkl

# IDE設定
.vscode/
.idea/

# OS関連
.DS_Store
Thumbs.db
\end{lstlisting}

\subsection{改行コードの問題}
異なるOS間でプロジェクトを共有する際、改行コードの違いによる警告が表示されることがあります:

\begin{lstlisting}
warning: in the working copy of 'file.py', 
LF will be replaced by CRLF the next time Git touches it
\end{lstlisting}

この警告は以下の方法で対処できます:

\begin{lstlisting}
# 改行コード自動変換を無効化
git config core.autocrlf false

# またはグローバル設定
git config --global core.autocrlf false
\end{lstlisting}

\subsection{ローカルGitとGitHubの使い分け}
\begin{itemize}
    \item \textbf{ローカルGit}: 全ての変更履歴、大きなファイル、個人的な実験結果を管理
    \item \textbf{GitHub}: ソースコードのみを公開、データセットやモデルファイルは除外
\end{itemize}

この使い分けにより、コードの共有と個人データの保護を両立できます。

\subsection{環境管理の将来展望}

\subsubsection{Docker による環境コンテナ化}
長期的な開発では、Dockerによる環境のコンテナ化が重要になります:

\begin{lstlisting}[language=bash]
# Dockerfile例
FROM python:3.11-slim
WORKDIR /app
COPY requirements.txt .
RUN pip install -r requirements.txt
COPY . .
CMD ["python", "exefile/mn_cnn_main.py"]
\end{lstlisting}

\subsubsection{段階的な環境管理戦略}
\begin{enumerate}
    \item \textbf{学習期}: Git + .gitignore による基本管理
    \item \textbf{中級期}: Docker導入による環境統一
    \item \textbf{上級期}: Kubernetes + CI/CD による本格運用
\end{enumerate}

この段階的アプローチにより、学習コストを最小化しながら
本格的な機械学習システムへの発展が可能になります。

\section{Advanced PyTorch: torch.func による関数型プログラミング}

\subsection{torch.func とは}
PyTorch 2.0以降で導入された\texttt{torch.func}は、モデルのパラメータを
直接変更することなく、関数的にモデルを操作できる革新的な機能です。

\subsection{従来の方法の問題点}
従来のパラメータ変更は危険な副作用を持ちます:

\begin{lstlisting}[language=Python]
# 危険: モデルのパラメータを直接変更
for i, param in enumerate(model.parameters()):
    param.data = trained_params[i] + t * random_vector[i]

outputs = model(input)  # 変更されたパラメータで実行

# パラメータを手動で復元する必要がある
for i, param in enumerate(model.parameters()):
    param.data = trained_params[i]
\end{lstlisting}

\subsection{torch.func による安全な実装}
\texttt{functional\_call}を使用すると、元のモデルを変更せずに
異なるパラメータでモデルを実行できます:

\begin{lstlisting}[language=Python]
import torch.func

def compute_loss_with_perturbation(model, t, random_vector, 
                                 test_loader, criterion):
    # 1. 摂動パラメータを辞書形式で準備
    perturbed_params = {}
    for (name, param), rand_vec in zip(model.named_parameters(), 
                                      random_vector):
        perturbed_params[name] = param + t * rand_vec
    
    # 2. 元のモデルは変更せず、一時的にパラメータを適用
    model.eval()
    total_loss = 0.0
    with torch.no_grad():
        for images, labels in test_loader:
            # functional_call: 非破壊的なモデル実行
            outputs = torch.func.functional_call(
                model, perturbed_params, images)
            loss = criterion(outputs, labels)
            total_loss += loss.item()
    
    return total_loss / len(test_loader)
\end{lstlisting}

\subsection{実用例: 損失ランドスケープ解析}
機械学習の最適化過程を視覚化する損失ランドスケープ解析での使用例:

\begin{lstlisting}[language=Python]
def analyze_loss_landscape(model, test_loader, criterion):
    # ランダムな方向ベクトルを生成
    random_vector = [torch.randn_like(param) 
                    for param in model.parameters()]
    
    # t ∈ [-0.05, 0.05] の範囲で損失を計算
    t_range = torch.linspace(-0.05, 0.05, 50)
    loss_values = []
    
    for t in t_range:
        loss = compute_loss_with_perturbation(
            model, t, random_vector, test_loader, criterion)
        loss_values.append(loss)
    
    return t_range, loss_values

# 使用例
model = trained_cnn_model
t_vals, losses = analyze_loss_landscape(model, test_loader, criterion)

# 結果のプロット
import matplotlib.pyplot as plt
plt.plot(t_vals.numpy(), losses)
plt.xlabel('Parameter Perturbation (t)')
plt.ylabel('Loss')
plt.title('Loss Landscape around Optimum')
plt.show()
\end{lstlisting}

\subsection{torch.func の利点}
\begin{enumerate}
    \item \textbf{安全性}: 元のモデルパラメータが変更されない
    \item \textbf{並列処理}: 複数のパラメータセットを同時に試験可能
    \item \textbf{関数型}: 副作用のない純粋関数として動作
    \item \textbf{デバッグ性}: パラメータの復元忘れによるバグを防止
\end{enumerate}

この手法により、安全で効率的なモデル解析が可能になり、
機械学習研究における新しいアプローチが開けます。

\section{高級Python機能の解説}

\subsection{zip関数の活用}
\texttt{zip}関数は複数のイテラブルを同時に処理する強力な機能です:

\begin{lstlisting}[language=Python]
# 基本的なzip
names = ['Alice', 'Bob', 'Charlie']
ages = [25, 30, 35]

for name, age in zip(names, ages):
    print(f"{name}は{age}歳です")
# 出力: Aliceは25歳です, Bobは30歳です, Charlieは35歳です

# torch.funcでの応用例
for (name, param), rand_vec in zip(model.named_parameters(), random_vector):
    perturbed_params[name] = param + t * rand_vec
\end{lstlisting}

\subsubsection{zipの詳細動作}
\texttt{zip}は以下のように動作します:

\begin{lstlisting}[language=Python]
# zipの内部動作イメージ
list1 = [1, 2, 3]
list2 = ['a', 'b', 'c']
zipped = zip(list1, list2)

# zipは次のタプルを順次生成
# (1, 'a'), (2, 'b'), (3, 'c')

# リストに変換して確認
print(list(zip(list1, list2)))
# [(1, 'a'), (2, 'b'), (3, 'c')]
\end{lstlisting}

\subsection{辞書の作成と操作}

\subsubsection{辞書内包表記(Dictionary Comprehension)}
波括弧\texttt{\{\}}を使った効率的な辞書作成:

\begin{lstlisting}[language=Python]
# 基本的な辞書内包表記
squares = {x: x**2 for x in range(5)}
print(squares)  # {0: 0, 1: 1, 2: 4, 3: 9, 4: 16}

# 条件付き辞書内包表記
even_squares = {x: x**2 for x in range(10) if x % 2 == 0}
print(even_squares)  # {0: 0, 2: 4, 4: 16, 6: 36, 8: 64}

# torch.funcでの実際の使用例
trained_params = {name: param.data.clone() 
                 for name, param in model.named_parameters()}
\end{lstlisting}

\subsubsection{通常の辞書作成との比較}
\begin{lstlisting}[language=Python]
# 従来の方法(冗長)
trained_params = {}
for name, param in model.named_parameters():
    trained_params[name] = param.data.clone()

# 辞書内包表記(簡潔)
trained_params = {name: param.data.clone() 
                 for name, param in model.named_parameters()}
\end{lstlisting}

\subsection{ジェネレータ式とリスト内包表記}

\subsubsection{リスト内包表記}
角括弧\texttt{[]}を使った効率的なリスト作成:

\begin{lstlisting}[language=Python]
# 基本的なリスト内包表記
squares = [x**2 for x in range(5)]
print(squares)  # [0, 1, 4, 9, 16]

# torch.funcでの使用例
random_vector = [torch.randn_like(param.data) 
                for param in model.parameters()]

# 条件付きリスト内包表記
large_params = [param for param in model.parameters() 
               if param.numel() > 1000]
\end{lstlisting}

\subsubsection{従来の方法との比較}
\begin{lstlisting}[language=Python]
# 従来の方法
random_vector = []
for param in model.parameters():
    random_vector.append(torch.randn_like(param.data))

# リスト内包表記(Pythonic)
random_vector = [torch.randn_like(param.data) 
                for param in model.parameters()]
\end{lstlisting}

\subsection{enumerate関数の活用}
インデックスと値を同時に取得する便利な関数:

\begin{lstlisting}[language=Python]
# 基本的なenumerate
items = ['apple', 'banana', 'cherry']
for i, item in enumerate(items):
    print(f"{i}: {item}")
# 出力: 0: apple, 1: banana, 2: cherry

# 機械学習での使用例(従来コード)
for i, param in enumerate(model.parameters()):
    param.data = trained_params[i] + t * random_vector[i]
\end{lstlisting}

\subsection{実践的な組み合わせ例}

\subsubsection{複雑なデータ処理の例}
これらの機能を組み合わせた実践例:

\begin{lstlisting}[language=Python]
# データの前処理例
data = [('Alice', 85), ('Bob', 92), ('Charlie', 78)]

# 辞書内包表記 + 条件付きフィルタリング
high_scorers = {name: score for name, score in data if score >= 80}
print(high_scorers)  # {'Alice': 85, 'Bob': 92}

# リスト内包表記 + enumerate + zip
names = ['Alice', 'Bob', 'Charlie']
scores = [85, 92, 78]
indexed_data = [(i, name, score) 
               for i, (name, score) in enumerate(zip(names, scores))]
print(indexed_data)
# [(0, 'Alice', 85), (1, 'Bob', 92), (2, 'Charlie', 78)]
\end{lstlisting}

\subsubsection{torch.funcコードの詳細解説}
実際のコードを行ごとに解説:

\begin{lstlisting}[language=Python]
# 1. 空の辞書を作成
perturbed_params = {}

# 2. zip関数で2つのイテラブルを同時処理
#    model.named_parameters(): (名前, パラメータ)のタプルを生成
#    random_vector: ランダムベクトルのリスト
for (name, param), rand_vec in zip(model.named_parameters(), random_vector):
    # 3. 辞書に新しいキー・値ペアを追加
    #    キー: パラメータの名前(文字列)
    #    値: 元のパラメータ + 摂動
    perturbed_params[name] = param + t * rand_vec

# 結果: {'conv1.weight': tensor(...), 'conv1.bias': tensor(...), ...}
\end{lstlisting}

\subsection{Pythonらしいコードの書き方}
これらの機能を使うことで、より\textbf{Pythonic}なコードが書けます:

\begin{itemize}
    \item \textbf{可読性}: コードの意図が明確
    \item \textbf{簡潔性}: 少ない行数で同じ処理を実現
    \item \textbf{効率性}: C言語レベルで最適化された内部処理
    \item \textbf{保守性}: バグが入りにくい構造
\end{itemize}

これらの高級機能をマスターすることで、より効率的で美しい
機械学習コードが書けるようになります。

\end{document}
